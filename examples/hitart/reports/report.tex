% !Mode:: "TeX:UTF-8"
\documentclass[fontset=windows,toc=true,type=master,stage=opening,campus=shenzhen]{hithesisart}
% 此处选项中不要有空格
%%%%%%%%%%%%%%%%%%%%%%%%%%%%%%%%%%%%%%%%%%%%%%%%%%%%%%%%%%%%%%%%%%%%%%%%%%%%%%%%
% 必填选项
% type=doctor|master|bachelor
% stage=opening|midterm
%%%%%%%%%%%%%%%%%%%%%%%%%%%%%%%%%%%%%%%%%%%%%%%%%%%%%%%%%%%%%%%%%%%%%%%%%%%%%%%%
% 选填选项(选填选项的缺省值已经尽可能满足了大多数需求,除非明确知道自己有什么
% 需求)
% campus=shenzhen|weihai|harbin
%   含义:校区选项,默认harbin
% fontset=windows|mac|ubuntu|fandol
%   含义:如果是安装了 windows 字体的 linux 系统,可以填写windows(win vista
%   以后 的字体)或 windowsold(vista 以前)。
%
%   如果是Mac系统,需要明确指定fontset=mac,由于各种各样的原因,多数情况是无法自
%   动识别mac系统字体,需要明确指定这个选项。注意,这时候使用的是mac系统中的宋
%   体,与windows中的宋体可能有点差别,虽然规范中没有明确规定使用那一种宋体,不
%   过按照窝工一贯的“规格严格,功夫到家”的风格,使用windows中的宋体肯定没毛病。
%   所以推荐Mac安装windows中的宋体。
%
%   fandol是一开源使用字体集,易于安装。这个选项放在这里是用来测试MacTeX/BasicTeX。
%%%%%%%%%%%%%%%%%%%%%%%%%%%%%%%%%%%%%%%%%%%%%%%%%%%%%%%%%%%%%%%%%%%%%%%%%%%%%%%%


\graphicspath{{figures/}}

\begin{document}

% !Mode:: "TeX:UTF-8"

\hitsetup{
  %******************************
  % 注意:
  %   1. 配置里面不要出现空行
  %   2. 不需要的配置信息可以删除
  %******************************
  ctitlecover={局部多孔质气体静压轴承关键技术的研究},%放在封面中使用,自由断行
  csubject={机械制造及其自动化},
  cauthor={于冬梅},
  cstudentid={9527},
  cclassid={9527},
  csupervisor={某某某教授},
  % 日期自动使用当前时间,若需指定按如下方式修改:
  %cdate={盘古开天地}
}

\makecover

% 威海校区本科生开题报告结构 ----------------------------------------
% !Mode:: "TeX:UTF-8"
\section{课题背景及研究的目的和意义}
\subsection{课题背景}
(正文  宋体小4号字,多倍行距值1.25,段前0行,段后0行。字数3000字以上。具体的撰写要符合哈尔滨工业大学本科生毕业论文撰写规范的书写规定。)\cite{hithesis2017}\inlinecite{cnproceed}
\subsection{研究的目的和意义}
\section{国内外在该方向的研究现状及分析}
\subsection{国外现状及分析}
\subsection{国内现状及分析}
\section{研究内容及拟解决的关键问题}
\subsection{研究内容}
\subsection{拟解决的关键问题}
\section{拟采取的研究方法和技术路线、进度安排、预期达到的目标}
\subsection{拟采取的研究方法和技术路线}
\subsection{进度安排}
\subsection{预期达到的目标}
\section{课题已具备和所需的条件}
\section{研究过程中可能遇到的困难和问题,解决的措施}
\section{参考文献}
\bibliographystyle{hithesis}
\bibliography{reference}

% Local Variables:
% TeX-master: "../mainart"
% TeX-engine: xetex
% End:
\makebackcover
% -------------------------------------------------------------

% 威海校区本科生中期报告结构 ----------------------------------------
% \section{论文工作是否按预期进行、目前已完成的研究工作及结果}
\subsection{论文工作是否按预期进行}
(正文  宋体小4号字,多倍行距值1.25,段前0行,段后0行。字数3000字以上。具体的撰写要符合哈尔滨工业大学本科生毕业论文撰写规范的书写规定。)
\subsection{目前已完成的研究工作及结果}
\section{后期拟完成的研究工作及进度安排}
\subsection{后期拟完成的研究工作}
\subsection{后期进度安排}
\section{存在的问题与困难}
\section{论文按时完成的可能性}
\section{参考文献}
\bibliographystyle{hithesis}
\bibliography{reference}

% Local Variables:
% TeX-master: "../mainart"
% TeX-engine: xetex
% End:
% \makebackcover
% -------------------------------------------------------------

% 哈尔滨校区本科生开题报告结构 --------------------------------------
% \section{课题来源及研究的目	的和意义}
(正文  宋体小4号字,行距1.25倍,段前0行,段后0行)
\section{国内外在该方向的研究现状及分析}
\section{主要研究内容}
\section{研究方案}
\section{进度安排,预期达到的目标}
\section{课题已具备和所需的条件、经费}
\section{研究过程中可能遇到的困难和问题,解决的措施}
\section{主要参考文献}
\bibliographystyle{hithesis}
\bibliography{reference}

% Local Variables:
% TeX-master: "../mainart"
% TeX-engine: xetex
% End:
% -------------------------------------------------------------

\end{document}

% Local Variables:
% TeX-engine: xetex
% End: